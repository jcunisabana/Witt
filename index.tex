\documentclass[spanish,]{article}
\usepackage{lmodern}
\usepackage{amssymb,amsmath}
\usepackage{ifxetex,ifluatex}
\usepackage{fixltx2e} % provides \textsubscript
\ifnum 0\ifxetex 1\fi\ifluatex 1\fi=0 % if pdftex
  \usepackage[T1]{fontenc}
  \usepackage[utf8]{inputenc}
\else % if luatex or xelatex
  \ifxetex
    \usepackage{mathspec}
  \else
    \usepackage{fontspec}
  \fi
  \defaultfontfeatures{Ligatures=TeX,Scale=MatchLowercase}
\fi
% use upquote if available, for straight quotes in verbatim environments
\IfFileExists{upquote.sty}{\usepackage{upquote}}{}
% use microtype if available
\IfFileExists{microtype.sty}{%
\usepackage{microtype}
\UseMicrotypeSet[protrusion]{basicmath} % disable protrusion for tt fonts
}{}
\usepackage[margin=1in]{geometry}
\usepackage{hyperref}
\hypersetup{unicode=true,
            pdftitle={SFC - Wittgenstein},
            pdfborder={0 0 0},
            breaklinks=true}
\urlstyle{same}  % don't use monospace font for urls
\ifnum 0\ifxetex 1\fi\ifluatex 1\fi=0 % if pdftex
  \usepackage[shorthands=off,main=spanish]{babel}
\else
  \usepackage{polyglossia}
  \setmainlanguage[]{spanish}
\fi
\usepackage{graphicx,grffile}
\makeatletter
\def\maxwidth{\ifdim\Gin@nat@width>\linewidth\linewidth\else\Gin@nat@width\fi}
\def\maxheight{\ifdim\Gin@nat@height>\textheight\textheight\else\Gin@nat@height\fi}
\makeatother
% Scale images if necessary, so that they will not overflow the page
% margins by default, and it is still possible to overwrite the defaults
% using explicit options in \includegraphics[width, height, ...]{}
\setkeys{Gin}{width=\maxwidth,height=\maxheight,keepaspectratio}
\IfFileExists{parskip.sty}{%
\usepackage{parskip}
}{% else
\setlength{\parindent}{0pt}
\setlength{\parskip}{6pt plus 2pt minus 1pt}
}
\setlength{\emergencystretch}{3em}  % prevent overfull lines
\providecommand{\tightlist}{%
  \setlength{\itemsep}{0pt}\setlength{\parskip}{0pt}}
\setcounter{secnumdepth}{0}
% Redefines (sub)paragraphs to behave more like sections
\ifx\paragraph\undefined\else
\let\oldparagraph\paragraph
\renewcommand{\paragraph}[1]{\oldparagraph{#1}\mbox{}}
\fi
\ifx\subparagraph\undefined\else
\let\oldsubparagraph\subparagraph
\renewcommand{\subparagraph}[1]{\oldsubparagraph{#1}\mbox{}}
\fi

%%% Use protect on footnotes to avoid problems with footnotes in titles
\let\rmarkdownfootnote\footnote%
\def\footnote{\protect\rmarkdownfootnote}

%%% Change title format to be more compact
\usepackage{titling}

% Create subtitle command for use in maketitle
\newcommand{\subtitle}[1]{
  \posttitle{
    \begin{center}\large#1\end{center}
    }
}

\setlength{\droptitle}{-2em}

  \title{SFC - Wittgenstein}
    \pretitle{\vspace{\droptitle}\centering\huge}
  \posttitle{\par}
    \author{}
    \preauthor{}\postauthor{}
    \date{}
    \predate{}\postdate{}
  
\usepackage{booktabs}
\usepackage{longtable}
\usepackage{array}
\usepackage{multirow}
\usepackage{wrapfig}
\usepackage{float}
\usepackage{colortbl}
\usepackage{pdflscape}
\usepackage{tabu}
\usepackage{threeparttable}
\usepackage{threeparttablex}
\usepackage[normalem]{ulem}
\usepackage{makecell}
\usepackage{xcolor}

\usepackage{fontspec}
\setmainfont{Adobe Jenson Pro}

\begin{document}
\maketitle

\subsection{Descripción del seminario}\label{descripcion-del-seminario}

Los escritos filosóficos de Wittgenstein ejercen una fascinación
especial en quien los lee con paciencia y cuidado. En este seminario,
estudiaremos dos: el \emph{Tractatus Logico-Philosophicus} y las
\emph{Investigaciones Filosóficas}.

El \emph{Tractactus} es una investigación lógica sobre lenguaje
enmarcada en la línea de Gottlob Frege y Bertrand Russell; es un texto
abstracto, técnico y poco amable con el lector, pero cuya inusual mezcla
de lógica, metafísica, y mística se presta para una multiplicidad de
interpretaciones y énfasis.

Las \emph{Investigaciones } es una investigación sobre el lenguaje y la
mente, rica en ejemplos concretos y descriptivos, aunque quizá un tanto
enmarañados y que no siempre tienen un propósito filosófico claro; es un
rechazo a las posiciones del \emph{Tractatus} y es la obra más
influyente de Wittgenstein.

El estudio de estas dos obras nos ofrece la oportunidad de ver el cambio
en la concepción filosófica de Wittgenstein y su particular forma de
pensar y hacer filosofía. En la primera parte del seminario,
estudiaremos el \emph{Tractatus} de la mano de Tomasini (2017), quien
nos ofrecerá una visión sinóptica del libro y con base en la cual
podremos realizar una aproximación propia al texto. En la segunda parte
del seminario, estudiaremos las \emph{Investigaciones} enfocándonos en
pasajes particulares de la obra, buscando entender conceptos centrales
como ``juego del lenguaje'', ``regla'' y ``representación perspicua''.

\textbf{Profesor}: \href{../index.html}{Juan Camilo Espejo-Serna}\\
\textbf{Horario y salón}: Martes, 10:00am - 1:00pm. Atelier 103

\textbf{Página web del curso}:
\url{http://jcunisabana.github.io/wittgenstein-2019}

\subsection{Objetivos}\label{objetivos}

\begin{itemize}
\item
  Leer críticamente, analizar e interpretar la propuesta filosófica
  temprana de Ludwig Wittgenstein.
\item
  Leer críticamente, analizar e interpretar la propuesta filosófica
  tardía de Ludwig Wittgenstein.
\item
  Planear y elaborar textos interpretativos y argumentativos con base en
  las teorías de Ludwig Wittgenstein.
\item
  Utilizar TIC para apoyar el estudio filosófico de la obra de Ludwig
  Wittgenstein
\end{itemize}

\subsection{Metodología}\label{metodologia}

\paragraph{\texorpdfstring{\textbf{Antes de la
sesión}}{Antes de la sesión}}\label{antes-de-la-sesion}

\begin{itemize}
\item
  Quienes estén a cargo de la presentación oral deberán preparar una
  exposición de 20-25 minutos de duración.
\item
  Todos los estudiantes deberán subir un control de lectura por tarde
  \textbf{75 horas} antes de la sesión.
\end{itemize}

\paragraph{\texorpdfstring{\textbf{Durante la
sesión}}{Durante la sesión}}\label{durante-la-sesion}

\begin{itemize}
\item
  Quienes esten a cargo de la presentación oral deberán realizar su
  exposición. Pueden, por ejemplo, escribir un texto para ser leído, o
  preparar diapositivas para guiar su exposición o utilizar el tablero
  como apoyo. Es su decisión pero deben tener en cuenta que el objetivo
  es lograr presentar y explicar el texto correspondiente.
\item
  Todos deben atender con cuidado a la exposición y formular preguntas
  al respecto. Revisen si entienden la exposición y si están de acuerdo;
  pregunten por las relaciones con los temas anteriormente expuestos.
\end{itemize}

\begin{center}\rule{0.5\linewidth}{\linethickness}\end{center}

\subsection{Plan semanal}\label{plan-semanal}

\subsubsection{Martes 22 Enero}\label{martes-22-enero}

\textbf{Semana 1}

Vida y obra de Ludwig Wittgenstein. Presentación del programa y las
reglas de juego del seminario.

\emph{Actividades}

Presentación a cargo del profesor: las introducciones del TLP y las IF.
Taller de filosofía por recortes. Taller sobre el programa.

\emph{Para la próxima}

Leer \textbf{todo} el \emph{Tractatus Logico-Philosophicus}.

\begin{center}\rule{0.5\linewidth}{\linethickness}\end{center}

\subsubsection{Martes 29 Enero}\label{martes-29-enero}

\textbf{Semana 2}

La multiplicidad de interpretaciones del \emph{TLP}.

\emph{Actividades}

Taller sobre el orden del TLP. Presentación a cargo del profesor.

\emph{Para la próxima}

Leer Tomasini, Cap 2 y Tomasini, Cap 3. Control de lectura.

\begin{center}\rule{0.5\linewidth}{\linethickness}\end{center}

\subsubsection{Martes 05 Febrero}\label{martes-05-febrero}

\textbf{Semana 3}

Ontología y lógica

\emph{Actividades}

Presentación de Tomasini, Cap 2 a cargo de (Por definir).Presentación de
Tomasini, Cap 3 a cargo de (Por definir).

\emph{Para la próxima}

Leer Tomasini, Cap 4 y Tomasini, Cap 5. Control de lectura.

\begin{center}\rule{0.5\linewidth}{\linethickness}\end{center}

\subsubsection{Martes 12 Febrero}\label{martes-12-febrero}

\textbf{Semana 4}

Lógica, matemáticas y ciencia

\emph{Actividades}

Presentación de Tomasini, Cap 4 a cargo de (Por definir).Presentación de
Tomasini, Cap 5 a cargo de (Por definir).

\emph{Para la próxima}

Leer Tomasini, Cap 6 y Tomasini, Cap 7. Control de lectura.

\begin{center}\rule{0.5\linewidth}{\linethickness}\end{center}

\subsubsection{Martes 19 Febrero}\label{martes-19-febrero}

\textbf{Semana 5}

Solipsismo, misticismo y filosofía

\emph{Actividades}

Presentación de Tomasini, Cap 6 a cargo de (Por definir).Presentación de
Tomasini, Cap 7 a cargo de (Por definir).Presentación de Tomasini, Cap 8
a cargo de (Por definir).

\emph{Para la próxima}

Segunda versión del taller de filosofía por recortes.

\begin{center}\rule{0.5\linewidth}{\linethickness}\end{center}

\subsubsection{Martes 26 Febrero}\label{martes-26-febrero}

\textbf{Semana 6}

Interpretaciones propias del \emph{TLP}

\emph{Actividades}

Auto-evaluación y co-evaluación del taller de recortes.

\emph{Para la próxima}

Leer IF Parte I, §§1-88. Control de lectura

\begin{center}\rule{0.5\linewidth}{\linethickness}\end{center}

\subsubsection{Martes 05 Marzo}\label{martes-05-marzo}

\textbf{Semana 7}

Definición ostensiva

\emph{Actividades}

Presentación de IF Parte I, §§1-88 a cargo de (Por definir).

\emph{Para la próxima}

Leer IF Parte I, §§ 89-133. Control de lectura

\begin{center}\rule{0.5\linewidth}{\linethickness}\end{center}

\subsubsection{Martes 12 Marzo}\label{martes-12-marzo}

\textbf{Semana 8}

La noción de filosofía

\emph{Actividades}

Presentación de IF Parte I, §§ 89-133 a cargo de (Por definir).

\emph{Para la próxima}

Leer IF Parte I, §§ 134 - 242. Control de lectura

\begin{center}\rule{0.5\linewidth}{\linethickness}\end{center}

\subsubsection{Martes 19 Marzo}\label{martes-19-marzo}

\textbf{Semana 9}

Seguimiento de reglas

\emph{Actividades}

Presentación de IF Parte I, §§ 134 - 242 a cargo de (Por definir).

\emph{Para la próxima}

Leer IF Parte I, §§ 243 - 275. Control de lectura

\begin{center}\rule{0.5\linewidth}{\linethickness}\end{center}

\subsubsection{Martes 26 Marzo}\label{martes-26-marzo}

\textbf{Semana 10}

Lenguaje privado

\emph{Actividades}

Presentación de IF Parte I, §§ 243 - 275 a cargo de (Por definir).

\emph{Para la próxima}

Leer IF Parte I, §§276 - 307. Control de lectura

\begin{center}\rule{0.5\linewidth}{\linethickness}\end{center}

\subsubsection{Martes 02 Abril}\label{martes-02-abril}

\textbf{Semana 11}

Lo interno y lo externo.

\emph{Actividades}

Presentación de IF Parte I, §§276 - 307 a cargo de (Por definir).

\emph{Para la próxima}

Leer IF Parte II, 11. Control de lectura

\begin{center}\rule{0.5\linewidth}{\linethickness}\end{center}

\subsubsection{Martes 09 Abril}\label{martes-09-abril}

\textbf{Semana 12}

Ver y ver como

\emph{Actividades}

Presentación de IF Parte II, 11 a cargo de (Por definir).

\emph{Para la próxima}

Escribir \textbf{dos páginas} en respuesta a alguna de las preguntas
pre-establecidas.

\begin{center}\rule{0.5\linewidth}{\linethickness}\end{center}

\subsubsection{Martes 16 Abril}\label{martes-16-abril}

\textbf{Semana 13}

NO HAY SESIÓN (Semana Santa)

\emph{Actividades}

--

\emph{Para la próxima}

Leer (Comentarista). Control de lectura

\begin{center}\rule{0.5\linewidth}{\linethickness}\end{center}

\subsubsection{Martes 23 Abril}\label{martes-23-abril}

\textbf{Semana 14}

(Por definir)

\emph{Actividades}

Presentación de (Comentarista) a cargo de (Por definir).

\emph{Para la próxima}

Leer (Comentarista). Control de lectura

\begin{center}\rule{0.5\linewidth}{\linethickness}\end{center}

\subsubsection{Martes 30 Abril}\label{martes-30-abril}

\textbf{Semana 15}

(Por definir)

\emph{Actividades}

Presentación de (Comentarista) a cargo de (Por definir).

\emph{Para la próxima}

Leer (Comentarista). Control de lectura

\begin{center}\rule{0.5\linewidth}{\linethickness}\end{center}

\subsubsection{Martes 07 Mayo}\label{martes-07-mayo}

\textbf{Semana 16}

(Por definir)

\emph{Actividades}

Presentación de (Comentarista) a cargo de (Por definir).

\emph{Para la próxima}

Leer (Comentarista). Control de lectura

\begin{center}\rule{0.5\linewidth}{\linethickness}\end{center}

\subsubsection{Martes 14 Mayo}\label{martes-14-mayo}

\textbf{Semana 17}

(Por definir)

\emph{Actividades}

Presentación de (Comentarista) a cargo de (Por definir).

\emph{Para la próxima}

Revisión final del ensayo final.

\begin{center}\rule{0.5\linewidth}{\linethickness}\end{center}

\subsubsection{Martes 21 Mayo}\label{martes-21-mayo}

\textbf{Semana 18}

\emph{Actividades}

Entrega de ensayo final

\subsection{Evaluación}\label{evaluacion}

\paragraph{\texorpdfstring{\textbf{Talleres}}{Talleres}}\label{talleres}

Actividades para realizar en casa y en clase que servirán de pasos
intermedios en el aprendizaje. En clase se ofrecerán instrucciones más
precisas cuando llegue el momento de cada taller.

\paragraph{\texorpdfstring{\textbf{Presentación
oral}}{Presentación oral}}\label{presentacion-oral}

Extensión: entre 20 y 25 minutos.

La presentación debe 1) tener una reseña crítica del texto asignado y 2)
ofrecer puntos para la discusión. Ustedes son los encargados de la
sesión durante el tiempo asignado; es su deber coordinar la dinámica
durante ese tiempo.

\paragraph{\texorpdfstring{\textbf{Control de
lectura}}{Control de lectura}}\label{control-de-lectura}

Extensión: entre 400 y 1000 palabras.

Para cada lectura asignada, los estudiantes deben escribir un texto
corto con la tesis principal, tres afirmaciones/presuposiciones del
texto y tres preguntas/desafíos al texto.

Los controles deberán ser subidos a la plataforma virtual a más tardar
\textbf{75 horas} antes de la sesión. Todos los estudiantes empiezan con
5.0 en esta nota; por cada vez que no se participe dentro del rango de
tiempo especificado, la nota será disminuida de acuerdo con los
siguientes parámetros: primera vez: -0.5; segunda vez: -1.0; tercera
vez: -1.5; cuarta vez: -2.0.

Todos tienen un control de lectura ``de gracia''. Es decir, pueden dejar
de entregar uno sin problema; el primer control de lectura que les falte
no cuenta. Por ejemplo, si no entregan el control de lectura de la
sesión en la que tienen que hacer la presentación y entregan todos los
demás, su nota igual queda en 5.0.

\paragraph{\texorpdfstring{\textbf{Ensayo
argumentativo}}{Ensayo argumentativo}}\label{ensayo-argumentativo}

Extensión: entre 2000 y 3500 palabras.

Un texto argumentativo de en donde se responda a una pregunta. Se debe
hacer uso de literatura secundaria de acuerdo con las normas de
citación. En clase se ofrecerán las instrucciones más precisas cuando
llegue el momento.

\paragraph{\texorpdfstring{\textbf{Calificación}}{Calificación}}\label{calificacion}

\begin{tabular}{|c|c|c|}
\hline
\textbf{Actividad} & \textbf{\%} & \textbf{Corte}\\
\hline
Taller sobre el programa & 3\% & 1\\
\hline
Presentación oral & 6\% & 1\\
\hline
Auto-evaluación filosofía por recortes & 3\% & 1\\
\hline
Co-evaluación filosofía por recortes & 3\% & 1\\
\hline
Controles de lectura & 15\% & 1\\
\hline
Controles de lectura & 15\% & 2\\
\hline
Presentación oral & 15\% & 2\\
\hline
Primer borrador ensayo final & 8\% & 3\\
\hline
Controles de lectura & 8\% & 3\\
\hline
Ensayo final & 24\% & 3\\
\hline
\end{tabular}

\textbf{Toda} entrega tarde injustificada verá la nota disminuida en 0.5
unidades por cada día tarde. No haber entregado antes de la hora
acordada equivale a entregar un día tarde.


\end{document}
